% Options for packages loaded elsewhere
\PassOptionsToPackage{unicode}{hyperref}
\PassOptionsToPackage{hyphens}{url}
\documentclass[
]{article}
\usepackage{xcolor}
\usepackage[margin=1in]{geometry}
\usepackage{amsmath,amssymb}
\setcounter{secnumdepth}{-\maxdimen} % remove section numbering
\usepackage{iftex}
\ifPDFTeX
  \usepackage[T1]{fontenc}
  \usepackage[utf8]{inputenc}
  \usepackage{textcomp} % provide euro and other symbols
\else % if luatex or xetex
  \usepackage{unicode-math} % this also loads fontspec
  \defaultfontfeatures{Scale=MatchLowercase}
  \defaultfontfeatures[\rmfamily]{Ligatures=TeX,Scale=1}
\fi
\usepackage{lmodern}
\ifPDFTeX\else
  % xetex/luatex font selection
\fi
% Use upquote if available, for straight quotes in verbatim environments
\IfFileExists{upquote.sty}{\usepackage{upquote}}{}
\IfFileExists{microtype.sty}{% use microtype if available
  \usepackage[]{microtype}
  \UseMicrotypeSet[protrusion]{basicmath} % disable protrusion for tt fonts
}{}
\makeatletter
\@ifundefined{KOMAClassName}{% if non-KOMA class
  \IfFileExists{parskip.sty}{%
    \usepackage{parskip}
  }{% else
    \setlength{\parindent}{0pt}
    \setlength{\parskip}{6pt plus 2pt minus 1pt}}
}{% if KOMA class
  \KOMAoptions{parskip=half}}
\makeatother
\usepackage{color}
\usepackage{fancyvrb}
\newcommand{\VerbBar}{|}
\newcommand{\VERB}{\Verb[commandchars=\\\{\}]}
\DefineVerbatimEnvironment{Highlighting}{Verbatim}{commandchars=\\\{\}}
% Add ',fontsize=\small' for more characters per line
\usepackage{framed}
\definecolor{shadecolor}{RGB}{248,248,248}
\newenvironment{Shaded}{\begin{snugshade}}{\end{snugshade}}
\newcommand{\AlertTok}[1]{\textcolor[rgb]{0.94,0.16,0.16}{#1}}
\newcommand{\AnnotationTok}[1]{\textcolor[rgb]{0.56,0.35,0.01}{\textbf{\textit{#1}}}}
\newcommand{\AttributeTok}[1]{\textcolor[rgb]{0.13,0.29,0.53}{#1}}
\newcommand{\BaseNTok}[1]{\textcolor[rgb]{0.00,0.00,0.81}{#1}}
\newcommand{\BuiltInTok}[1]{#1}
\newcommand{\CharTok}[1]{\textcolor[rgb]{0.31,0.60,0.02}{#1}}
\newcommand{\CommentTok}[1]{\textcolor[rgb]{0.56,0.35,0.01}{\textit{#1}}}
\newcommand{\CommentVarTok}[1]{\textcolor[rgb]{0.56,0.35,0.01}{\textbf{\textit{#1}}}}
\newcommand{\ConstantTok}[1]{\textcolor[rgb]{0.56,0.35,0.01}{#1}}
\newcommand{\ControlFlowTok}[1]{\textcolor[rgb]{0.13,0.29,0.53}{\textbf{#1}}}
\newcommand{\DataTypeTok}[1]{\textcolor[rgb]{0.13,0.29,0.53}{#1}}
\newcommand{\DecValTok}[1]{\textcolor[rgb]{0.00,0.00,0.81}{#1}}
\newcommand{\DocumentationTok}[1]{\textcolor[rgb]{0.56,0.35,0.01}{\textbf{\textit{#1}}}}
\newcommand{\ErrorTok}[1]{\textcolor[rgb]{0.64,0.00,0.00}{\textbf{#1}}}
\newcommand{\ExtensionTok}[1]{#1}
\newcommand{\FloatTok}[1]{\textcolor[rgb]{0.00,0.00,0.81}{#1}}
\newcommand{\FunctionTok}[1]{\textcolor[rgb]{0.13,0.29,0.53}{\textbf{#1}}}
\newcommand{\ImportTok}[1]{#1}
\newcommand{\InformationTok}[1]{\textcolor[rgb]{0.56,0.35,0.01}{\textbf{\textit{#1}}}}
\newcommand{\KeywordTok}[1]{\textcolor[rgb]{0.13,0.29,0.53}{\textbf{#1}}}
\newcommand{\NormalTok}[1]{#1}
\newcommand{\OperatorTok}[1]{\textcolor[rgb]{0.81,0.36,0.00}{\textbf{#1}}}
\newcommand{\OtherTok}[1]{\textcolor[rgb]{0.56,0.35,0.01}{#1}}
\newcommand{\PreprocessorTok}[1]{\textcolor[rgb]{0.56,0.35,0.01}{\textit{#1}}}
\newcommand{\RegionMarkerTok}[1]{#1}
\newcommand{\SpecialCharTok}[1]{\textcolor[rgb]{0.81,0.36,0.00}{\textbf{#1}}}
\newcommand{\SpecialStringTok}[1]{\textcolor[rgb]{0.31,0.60,0.02}{#1}}
\newcommand{\StringTok}[1]{\textcolor[rgb]{0.31,0.60,0.02}{#1}}
\newcommand{\VariableTok}[1]{\textcolor[rgb]{0.00,0.00,0.00}{#1}}
\newcommand{\VerbatimStringTok}[1]{\textcolor[rgb]{0.31,0.60,0.02}{#1}}
\newcommand{\WarningTok}[1]{\textcolor[rgb]{0.56,0.35,0.01}{\textbf{\textit{#1}}}}
\usepackage{graphicx}
\makeatletter
\newsavebox\pandoc@box
\newcommand*\pandocbounded[1]{% scales image to fit in text height/width
  \sbox\pandoc@box{#1}%
  \Gscale@div\@tempa{\textheight}{\dimexpr\ht\pandoc@box+\dp\pandoc@box\relax}%
  \Gscale@div\@tempb{\linewidth}{\wd\pandoc@box}%
  \ifdim\@tempb\p@<\@tempa\p@\let\@tempa\@tempb\fi% select the smaller of both
  \ifdim\@tempa\p@<\p@\scalebox{\@tempa}{\usebox\pandoc@box}%
  \else\usebox{\pandoc@box}%
  \fi%
}
% Set default figure placement to htbp
\def\fps@figure{htbp}
\makeatother
\setlength{\emergencystretch}{3em} % prevent overfull lines
\providecommand{\tightlist}{%
  \setlength{\itemsep}{0pt}\setlength{\parskip}{0pt}}
\usepackage{bookmark}
\IfFileExists{xurl.sty}{\usepackage{xurl}}{} % add URL line breaks if available
\urlstyle{same}
\hypersetup{
  pdftitle={Project 6},
  pdfauthor={Justin Williams},
  hidelinks,
  pdfcreator={LaTeX via pandoc}}

\title{Project 6}
\author{Justin Williams}
\date{2024-11-22}

\begin{document}
\maketitle

\begin{Shaded}
\begin{Highlighting}[]
\CommentTok{\# Load necessary libraries}
\FunctionTok{library}\NormalTok{(readr)}
\FunctionTok{library}\NormalTok{(dplyr)}
\end{Highlighting}
\end{Shaded}

\begin{verbatim}
## 
## Attaching package: 'dplyr'
\end{verbatim}

\begin{verbatim}
## The following objects are masked from 'package:stats':
## 
##     filter, lag
\end{verbatim}

\begin{verbatim}
## The following objects are masked from 'package:base':
## 
##     intersect, setdiff, setequal, union
\end{verbatim}

\begin{Shaded}
\begin{Highlighting}[]
\CommentTok{\# Use the filter function from the stats package}
\CommentTok{\#stats::filter()}
\CommentTok{\# Use the intersect function from the base package}
\CommentTok{\#base::intersect()}

\CommentTok{\# Load the data}
\NormalTok{HRData }\OtherTok{\textless{}{-}} \FunctionTok{read\_csv}\NormalTok{(}\StringTok{"/Users/justinwilliams/Code/9050advresearch/Project 6/HRData.csv"}\NormalTok{)}
\end{Highlighting}
\end{Shaded}

\begin{verbatim}
## Rows: 311 Columns: 40
\end{verbatim}

\begin{verbatim}
## -- Column specification --------------------------------------------------------
## Delimiter: ","
## chr (20): Employee_Name, Position, State, DOB, Sex, MaritalDesc, CitizenDesc...
## dbl (20): EmpID, MarriedID, MaritalStatusID, GenderID, EmpStatusID, DeptID, ...
## 
## i Use `spec()` to retrieve the full column specification for this data.
## i Specify the column types or set `show_col_types = FALSE` to quiet this message.
\end{verbatim}

\section{1. Run a multiple regression analysis in which you predict
PerfScoreID from EmpSatisfaction, EngagementSurvey, and Tenure. Provide
a summary of this analysis like what you would find in a journal
article. Be sure to provide a table of results AND a written summary of
the results in your
response.}\label{run-a-multiple-regression-analysis-in-which-you-predict-perfscoreid-from-empsatisfaction-engagementsurvey-and-tenure.-provide-a-summary-of-this-analysis-like-what-you-would-find-in-a-journal-article.-be-sure-to-provide-a-table-of-results-and-a-written-summary-of-the-results-in-your-response.}

\begin{Shaded}
\begin{Highlighting}[]
\CommentTok{\# Calculate Tenure}
\NormalTok{HRData }\OtherTok{\textless{}{-}}\NormalTok{ HRData }\SpecialCharTok{\%\textgreater{}\%}
    \FunctionTok{mutate}\NormalTok{(}\AttributeTok{DateofHire =} \FunctionTok{as.Date}\NormalTok{(DateofHire, }\AttributeTok{format =} \StringTok{"\%m/\%d/\%Y"}\NormalTok{),}
           \AttributeTok{DateofTermination =} \FunctionTok{as.Date}\NormalTok{(DateofTermination, }\AttributeTok{format =} \StringTok{"\%m/\%d/\%Y"}\NormalTok{),}
           \AttributeTok{CurrentDate =} \FunctionTok{Sys.Date}\NormalTok{(),}
           \AttributeTok{Tenure =} \FunctionTok{ifelse}\NormalTok{(}\FunctionTok{is.na}\NormalTok{(DateofTermination), }
                           \FunctionTok{as.numeric}\NormalTok{(}\FunctionTok{difftime}\NormalTok{(CurrentDate, DateofHire, }\AttributeTok{units =} \StringTok{"days"}\NormalTok{)) }\SpecialCharTok{/} \FloatTok{365.25}\NormalTok{,}
                           \FunctionTok{as.numeric}\NormalTok{(}\FunctionTok{difftime}\NormalTok{(DateofTermination, DateofHire, }\AttributeTok{units =} \StringTok{"days"}\NormalTok{)) }\SpecialCharTok{/} \FloatTok{365.25}\NormalTok{))}
\CommentTok{\# Fit the multiple regression model}
\NormalTok{model }\OtherTok{\textless{}{-}} \FunctionTok{lm}\NormalTok{(PerfScoreID }\SpecialCharTok{\textasciitilde{}}\NormalTok{ EmpSatisfaction }\SpecialCharTok{+}\NormalTok{ EngagementSurvey }\SpecialCharTok{+}\NormalTok{ Tenure, }\AttributeTok{data =}\NormalTok{ HRData)}

\CommentTok{\# Summary of the model}
\FunctionTok{summary}\NormalTok{(model)}
\end{Highlighting}
\end{Shaded}

\begin{verbatim}
## 
## Call:
## lm(formula = PerfScoreID ~ EmpSatisfaction + EngagementSurvey + 
##     Tenure, data = HRData)
## 
## Residuals:
##      Min       1Q   Median       3Q      Max 
## -2.09618 -0.23836 -0.03649  0.24840  1.15086 
## 
## Coefficients:
##                   Estimate Std. Error t value Pr(>|t|)    
## (Intercept)      8.463e-01  1.738e-01   4.868 1.81e-06 ***
## EmpSatisfaction  1.348e-01  3.035e-02   4.442 1.24e-05 ***
## EngagementSurvey 3.749e-01  3.494e-02  10.730  < 2e-16 ***
## Tenure           4.908e-05  2.858e-05   1.717   0.0869 .  
## ---
## Signif. codes:  0 '***' 0.001 '**' 0.01 '*' 0.05 '.' 0.1 ' ' 1
## 
## Residual standard error: 0.4773 on 307 degrees of freedom
## Multiple R-squared:  0.3454, Adjusted R-squared:  0.339 
## F-statistic: 53.99 on 3 and 307 DF,  p-value: < 2.2e-16
\end{verbatim}

\begin{Shaded}
\begin{Highlighting}[]
\DocumentationTok{\#\# Create a table of results}
\NormalTok{results }\OtherTok{\textless{}{-}} \FunctionTok{summary}\NormalTok{(model)}\SpecialCharTok{$}\NormalTok{coefficients}
\NormalTok{results }\OtherTok{\textless{}{-}} \FunctionTok{as.data.frame}\NormalTok{(results)}
\FunctionTok{colnames}\NormalTok{(results) }\OtherTok{\textless{}{-}} \FunctionTok{c}\NormalTok{(}\StringTok{"Estimate"}\NormalTok{, }\StringTok{"Std. Error"}\NormalTok{, }\StringTok{"t value"}\NormalTok{, }\StringTok{"Pr(\textgreater{}|t|)"}\NormalTok{)}

\DocumentationTok{\#\# Print the table of results}
\FunctionTok{print}\NormalTok{(results)}
\end{Highlighting}
\end{Shaded}

\begin{verbatim}
##                      Estimate   Std. Error   t value     Pr(>|t|)
## (Intercept)      8.462855e-01 1.738443e-01  4.868065 1.806107e-06
## EmpSatisfaction  1.348302e-01 3.035167e-02  4.442268 1.244044e-05
## EngagementSurvey 3.749061e-01 3.494069e-02 10.729783 5.087448e-23
## Tenure           4.908324e-05 2.857864e-05  1.717480 8.689996e-02
\end{verbatim}

\subsection{Summary}\label{summary}

With an R² of 0.3454, the model demonstrates moderate explanatory power,
suggesting the predictors collectively capture meaningful variance in
PerfScoreID. EmpSatisfaction and EngagementSurvey are significant
predictors, showing their importance in performance evaluations.

The non-significance of Tenure highlights the point that not all
predictors contribute meaningfully to a model, meaning you must engage
in careful model refinement.

The results show the importance of employee satisfaction and engagement
in driving performance, offering actionable insights for organizational
interventions. You must interpret these findings with caution,
considering potential omitted variables and the unexplained variance.

In summary, this model highlights significant predictors while pointing
to areas for theoretical and practical improvement.

\subsection{Answers a, b, and c.}\label{answers-a-b-and-c.}

\subsubsection{a. What is the significance of the regression weights for
each predictor, and are they statistically
significant?}\label{a.-what-is-the-significance-of-the-regression-weights-for-each-predictor-and-are-they-statistically-significant}

The multiple regression analysis was conducted to predict PerfScoreID
based on EmpSatisfaction, EngagementSurvey, and Tenure. The model was
statistically significant, F(3, 307) = 53.99, p \textless{} 0.05,
explaining 34.54\% of the variance in PerfScoreID (R² = 0.3454). Below
are the regression weights for each predictor:

EmpSatisfaction * Estimate = 0.13, Std. Error = 0.03, t-value = 4.44, p
\textless{} 0.001. * EmpSatisfaction is a statistically significant
predictor of PerfScoreID, indicating that higher satisfaction is
associated with higher performance scores.

EngagementSurvey * Estimate = 0.37, Std. Error = 0.03, t-value = 10.73,
p \textless{} 0.001. * EngagementSurvey is also a statistically
significant predictor, suggesting that better engagement survey scores
are linked to higher performance scores.

Tenure * Estimate = 0.00, Std. Error = 0.00, t-value = 1.72, p = 0.0869.
* While Tenure has a small positive relationship with PerfScoreID, it is
not statistically significant at the level of 0.05.

\subsubsection{a. What is the significance of the regression weights for
each predictor and are they statistically
significant?}\label{a.-what-is-the-significance-of-the-regression-weights-for-each-predictor-and-are-they-statistically-significant-1}

The significance of the regression weights for each predictor is as
follows:

\begin{itemize}
\tightlist
\item
  EmpSatisfaction: Estimate = 0.13, Std. Error = 0.03, t value = 4.44, p
  \textless{} 0.001. This suggests that higher employee satisfaction is
  associated with higher performance scores, controlling for other
  variables.
\item
  EngagementSurvey: Estimate = 0.37, Std. Error = 0.03, t value = 10.73,
  p \textless{} 0.001. This indicates that higher engagement survey
  scores are strongly associated with higher performance scores.
\item
  Tenure: Estimate = 0, Std. Error = 0, t value = 1.72, p = 0.0869.
  Tenure does not significantly predict performance scores at the
  conventional significance level.
\end{itemize}

\subsubsection{b. What is the regression equation for the most efficient
model?}\label{b.-what-is-the-regression-equation-for-the-most-efficient-model}

The regression equation for the model is:

\begin{itemize}
\tightlist
\item
  PerfScoreID = 0.85 + 0.13 * EmpSatisfaction + 0.37 * EngagementSurvey
  + 0 * Tenure.
\end{itemize}

\subsubsection{c.~What percentage of variance in PerfScoreID is
explained by the
model?}\label{c.-what-percentage-of-variance-in-perfscoreid-is-explained-by-the-model}

The overall model was statistically significant, F(3, 307) = 53.99, p
\textless{} 0.05, explaining 34.54\% of the variance in PerfScoreID.
This indicates that the combination of predictors contributes
meaningfully to predicting the dependent variable.

\section{2. Repeat the analysis you ran in Question 1 in which you
predict PerfScoreID from EmpSatisfaction, EngagementSurvey, and Tenure
but first control for Department. Provide a summary of this analysis
like what you would find in a journal article. Be sure to provide a
table of results AND a written summary of the results in your
response.}\label{repeat-the-analysis-you-ran-in-question-1-in-which-you-predict-perfscoreid-from-empsatisfaction-engagementsurvey-and-tenure-but-first-control-for-department.-provide-a-summary-of-this-analysis-like-what-you-would-find-in-a-journal-article.-be-sure-to-provide-a-table-of-results-and-a-written-summary-of-the-results-in-your-response.}

\begin{Shaded}
\begin{Highlighting}[]
\DocumentationTok{\#\# Fit the multiple regression model with Department as a covariate}
\NormalTok{model\_step1 }\OtherTok{\textless{}{-}} \FunctionTok{lm}\NormalTok{(PerfScoreID }\SpecialCharTok{\textasciitilde{}}\NormalTok{ Department, }\AttributeTok{data =}\NormalTok{ HRData)}
\NormalTok{model\_step2 }\OtherTok{\textless{}{-}} \FunctionTok{lm}\NormalTok{(PerfScoreID }\SpecialCharTok{\textasciitilde{}}\NormalTok{ Department }\SpecialCharTok{+}\NormalTok{ EmpSatisfaction }\SpecialCharTok{+}\NormalTok{ EngagementSurvey }\SpecialCharTok{+}\NormalTok{ Tenure, }\AttributeTok{data =}\NormalTok{ HRData)}

\DocumentationTok{\#\# Summary of the models}
\NormalTok{summary\_step1 }\OtherTok{\textless{}{-}} \FunctionTok{summary}\NormalTok{(model\_step1)}
\NormalTok{summary\_step2 }\OtherTok{\textless{}{-}} \FunctionTok{summary}\NormalTok{(model\_step2)}
\end{Highlighting}
\end{Shaded}

\begin{Shaded}
\begin{Highlighting}[]
\DocumentationTok{\#\# Create tables of results}
\NormalTok{results\_step1 }\OtherTok{\textless{}{-}} \FunctionTok{as.data.frame}\NormalTok{(summary\_step1}\SpecialCharTok{$}\NormalTok{coefficients)}
\FunctionTok{colnames}\NormalTok{(results\_step1) }\OtherTok{\textless{}{-}} \FunctionTok{c}\NormalTok{(}\StringTok{"Estimate"}\NormalTok{, }\StringTok{"Std. Error"}\NormalTok{, }\StringTok{"t value"}\NormalTok{, }\StringTok{"Pr(\textgreater{}|t|)"}\NormalTok{)}

\NormalTok{results\_step2 }\OtherTok{\textless{}{-}} \FunctionTok{as.data.frame}\NormalTok{(summary\_step2}\SpecialCharTok{$}\NormalTok{coefficients)}
\FunctionTok{colnames}\NormalTok{(results\_step2) }\OtherTok{\textless{}{-}} \FunctionTok{c}\NormalTok{(}\StringTok{"Estimate"}\NormalTok{, }\StringTok{"Std. Error"}\NormalTok{, }\StringTok{"t value"}\NormalTok{, }\StringTok{"Pr(\textgreater{}|t|)"}\NormalTok{)}

\DocumentationTok{\#\# Print the tables of results}
\FunctionTok{print}\NormalTok{(results\_step1)}
\end{Highlighting}
\end{Shaded}

\begin{verbatim}
##                                     Estimate Std. Error       t value
## (Intercept)                     3.000000e+00  0.1962772  1.528451e+01
## DepartmentExecutive Office     -5.035592e-16  0.6206829 -8.112987e-16
## DepartmentIT/IS                 6.000000e-02  0.2132116  2.814106e-01
## DepartmentProduction           -2.870813e-02  0.2004587 -1.432122e-01
## DepartmentSales                -1.612903e-01  0.2229559 -7.234181e-01
## DepartmentSoftware Engineering  9.090909e-02  0.2646601  3.434938e-01
##                                    Pr(>|t|)
## (Intercept)                    1.494910e-39
## DepartmentExecutive Office     1.000000e+00
## DepartmentIT/IS                7.785863e-01
## DepartmentProduction           8.862172e-01
## DepartmentSales                4.699775e-01
## DepartmentSoftware Engineering 7.314637e-01
\end{verbatim}

\begin{Shaded}
\begin{Highlighting}[]
\FunctionTok{print}\NormalTok{(results\_step2)}
\end{Highlighting}
\end{Shaded}

\begin{verbatim}
##                                     Estimate   Std. Error     t value
## (Intercept)                     8.070708e-01 2.385241e-01  3.38360257
## DepartmentExecutive Office     -1.112967e-01 5.065934e-01 -0.21969627
## DepartmentIT/IS                 9.312904e-02 1.745347e-01  0.53358450
## DepartmentProduction            4.654892e-02 1.643086e-01  0.28330171
## DepartmentSales                -1.667016e-02 1.838105e-01 -0.09069207
## DepartmentSoftware Engineering  1.574024e-01 2.169516e-01  0.72551829
## EmpSatisfaction                 1.335070e-01 3.081265e-02  4.33286250
## EngagementSurvey                3.731895e-01 3.562790e-02 10.47464219
## Tenure                          5.028933e-05 2.935399e-05  1.71320252
##                                    Pr(>|t|)
## (Intercept)                    8.098340e-04
## DepartmentExecutive Office     8.262561e-01
## DepartmentIT/IS                5.940215e-01
## DepartmentProduction           7.771397e-01
## DepartmentSales                9.277974e-01
## DepartmentSoftware Engineering 4.686960e-01
## EmpSatisfaction                2.006305e-05
## EngagementSurvey               4.181577e-22
## Tenure                         8.770155e-02
\end{verbatim}

\begin{Shaded}
\begin{Highlighting}[]
\DocumentationTok{\#\# Calculate variance explained by Department in Step 1}
\NormalTok{variance\_explained\_step1 }\OtherTok{\textless{}{-}}\NormalTok{ summary\_step1}\SpecialCharTok{$}\NormalTok{r.squared }\SpecialCharTok{*} \DecValTok{100}

\CommentTok{\# Calculate change in R{-}square from Step 1 to Step 2}
\NormalTok{change\_in\_r\_squared }\OtherTok{\textless{}{-}}\NormalTok{ summary\_step2}\SpecialCharTok{$}\NormalTok{r.squared }\SpecialCharTok{{-}}\NormalTok{ summary\_step1}\SpecialCharTok{$}\NormalTok{r.squared}

\CommentTok{\# Calculate percentage of variance explained by the full model}
\NormalTok{variance\_explained\_full\_model }\OtherTok{\textless{}{-}}\NormalTok{ summary\_step2}\SpecialCharTok{$}\NormalTok{r.squared }\SpecialCharTok{*} \DecValTok{100}
\end{Highlighting}
\end{Shaded}

\subsection{Summary}\label{summary-1}

Department alone accounted for only 1.02\% of the variance in
PerfScoreID, showing it has limited predictive utility when used as a
standalone covariate.

Adding EmpSatisfaction, EngagementSurvey, and Tenure in Step 2 increased
the variance explained by 33.89 percentage points, showing the added
value of these predictors.

The full model explained 34.92\% of the variance in PerfScoreID,
reflecting its considerably stronger explanatory power compared to Step
1.

\subsection{Answers a, b, and c.}\label{answers-a-b-and-c.-1}

\subsubsection{a. How much variance did Department explain as a
covariate in Step 1 of your
Model?}\label{a.-how-much-variance-did-department-explain-as-a-covariate-in-step-1-of-your-model}

In Step 1, where only Department was included as a predictor, the model
explained 1.02\% of the variance in PerfScoreID (R² = 0.0102). While the
model was statistically significant (F(5, 305) = 0.63, p \textless{}
0.05), the amount of variance explained by Department is minimal,
showing that Department alone is not a strong predictor of PerfScoreID.

\subsubsection{b. What is the change in R-square when you go from Step 1
to Step
2?}\label{b.-what-is-the-change-in-r-square-when-you-go-from-step-1-to-step-2}

In Step 2, additional predictors (EmpSatisfaction, EngagementSurvey, and
Tenure) were added to the model. This resulted in a large increase in
explanatory power, with R² increasing from 0.0102 to 0.3492. The change
in R² was 0.3389, suggesting that the inclusion of these additional
predictors significantly improved the model's ability to explain
variance in PerfScoreID.

\subsubsection{c.~What percentage of variance in PerfScoreID is
explained by the full
model?}\label{c.-what-percentage-of-variance-in-perfscoreid-is-explained-by-the-full-model}

The Full Model in Step 2 explained 34.92\% of the variance in
PerfScoreID (R² = 0.3492). This is a large improvement over Step 1,
demonstrating that adding EmpSatisfaction, EngagementSurvey, and Tenure
substantially increased the predictive power of the model.

\section{3. Run a multiple regression analysis in which you predict
Absences from EmpSatisfaction, EngagementSurvey, and Tenure. Provide a
summary of this analysis like what you would find in a journal article.
Be sure to provide a table of results AND a written summary of the
results in your
response.}\label{run-a-multiple-regression-analysis-in-which-you-predict-absences-from-empsatisfaction-engagementsurvey-and-tenure.-provide-a-summary-of-this-analysis-like-what-you-would-find-in-a-journal-article.-be-sure-to-provide-a-table-of-results-and-a-written-summary-of-the-results-in-your-response.}

\begin{Shaded}
\begin{Highlighting}[]
\CommentTok{\# Fit the multiple regression model to predict Absences}
\NormalTok{model\_absences }\OtherTok{\textless{}{-}} \FunctionTok{lm}\NormalTok{(Absences }\SpecialCharTok{\textasciitilde{}}\NormalTok{ EmpSatisfaction }\SpecialCharTok{+}\NormalTok{ EngagementSurvey }\SpecialCharTok{+}\NormalTok{ Tenure, }\AttributeTok{data =}\NormalTok{ HRData)}

\CommentTok{\# Summary of the model}
\NormalTok{summary\_absences }\OtherTok{\textless{}{-}} \FunctionTok{summary}\NormalTok{(model\_absences)}
\end{Highlighting}
\end{Shaded}

\begin{Shaded}
\begin{Highlighting}[]
\CommentTok{\# Create a table of results}
\NormalTok{results\_absences }\OtherTok{\textless{}{-}} \FunctionTok{as.data.frame}\NormalTok{(summary\_absences}\SpecialCharTok{$}\NormalTok{coefficients)}
\FunctionTok{colnames}\NormalTok{(results\_absences) }\OtherTok{\textless{}{-}} \FunctionTok{c}\NormalTok{(}\StringTok{"Estimate"}\NormalTok{, }\StringTok{"Std. Error"}\NormalTok{, }\StringTok{"t value"}\NormalTok{, }\StringTok{"Pr(\textgreater{}|t|)"}\NormalTok{)}

\CommentTok{\# Print the table of results}
\FunctionTok{print}\NormalTok{(results\_absences)}
\end{Highlighting}
\end{Shaded}

\begin{verbatim}
##                       Estimate   Std. Error   t value     Pr(>|t|)
## (Intercept)       9.7203396201 2.1249537899  4.574377 6.939354e-06
## EmpSatisfaction   0.5136100198 0.3709979331  1.384401 1.672410e-01
## EngagementSurvey -0.1627691183 0.4270910525 -0.381111 7.033845e-01
## Tenure           -0.0006058401 0.0003493257 -1.734313 8.386630e-02
\end{verbatim}

\begin{Shaded}
\begin{Highlighting}[]
\CommentTok{\# Calculate percentage of variance explained by the model}
\NormalTok{variance\_explained\_absences }\OtherTok{\textless{}{-}}\NormalTok{ summary\_absences}\SpecialCharTok{$}\NormalTok{r.squared }\SpecialCharTok{*} \DecValTok{100}
\end{Highlighting}
\end{Shaded}

\subsection{Summary}\label{summary-2}

While it's important to interpret predictor significance and understand
its contribution to the model, this analysis highlights that predictors
such as EmpSatisfaction, EngagementSurvey, and Tenure do not
significantly influence Absences individually, despite the model
reaching statistical significance overall. This discrepancy underscores
the importance of distinguishing between overall model significance and
individual predictor contributions.

The low R² value suggests that much of the variability in Absences
remains unexplained. Practical significance reminds us that even
statistically significant models can lack real world utility if their
explanatory power is minimal.

Alternative predictors or interaction effects might improve the model's
predictive capability, as the current variables appear unable to
meaningfully explain variations in Absences.

The findings here can guide future analyses to identify more robust
predictors of Absences.

\subsection{Answers a, b, and c.}\label{answers-a-b-and-c.-2}

\subsubsection{a. What is the significance of the regression weights for
each predictor, and are they statistically
significant?}\label{a.-what-is-the-significance-of-the-regression-weights-for-each-predictor-and-are-they-statistically-significant-2}

The multiple regression model examined EmpSatisfaction,
EngagementSurvey, and Tenure as predictors of Absences. The significance
of the regression weights for each predictor is summarized below.

\begin{itemize}
\tightlist
\item
  EmpSatisfaction

  \begin{itemize}
  \tightlist
  \item
    Estimate = 0.51, Std. Error = 0.37, t-value = 1.38, p = 0.1672.
  \item
    This suggests a positive relationship between EmpSatisfaction and
    Absences; however, the p-value indicates that this predictor is not
    statistically significant.
  \end{itemize}
\item
  EngagementSurvey

  \begin{itemize}
  \tightlist
  \item
    Estimate = -0.16, Std. Error = 0.43, t-value = -0.38, p = 0.7034.
  \item
    While the negative estimate suggests a potential inverse
    relationship between EngagementSurvey and Absences, the high p-value
    shows that it is not statistically significant.
  \end{itemize}
\item
  Tenure

  \begin{itemize}
  \tightlist
  \item
    Estimate = -0.0006, Std. Error = 0.0003, t-value = -1.73, p =
    0.0839.
  \item
    The small negative coefficient indicates a weak inverse relationship
    between Tenure and Absences. With a p-value close to 0.05, Tenure
    approaches but does not meet the threshold for statistical
    significance.
  \end{itemize}
\end{itemize}

None of the predictors in the model were statistically significant at
the 0.05 level. This suggests that while these variables collectively
contribute to the model, their individual contributions to predicting
Absences are limited.

\subsubsection{b. What is the regression equation for the most efficient
model?}\label{b.-what-is-the-regression-equation-for-the-most-efficient-model-1}

The regression equation for the most efficient model is:

\begin{itemize}
\tightlist
\item
  Absences = 9.72 + 0.51 ⋅ EmpSatisfaction − 0.16 ⋅ EngagementSurvey −
  0.0006 ⋅ Tenure
\end{itemize}

This equation shows that Absences is predicted to increase with higher
EmpSatisfaction and decrease with higher EngagementSurvey and Tenure,
though these effects are not statistically significant.

\subsubsection{c.~What percentage of variance in Absences is explained
by the
model?}\label{c.-what-percentage-of-variance-in-absences-is-explained-by-the-model}

The model explained 1.58\% of the variance in Absences (R² = 0.0158).
Although the model itself was statistically significant (F(3, 307) =
1.65, p \textless{} 0.05), the low R² value indicates that the
predictors only account for a small fraction of the variability in
Absences.

\section{4. Repeat the analysis you ran in Question 3 in which you
predict Absences from EmpSatisfaction, EngagementSurvey, and Tenure but
first control for Sex. Provide a summary of this analysis like what you
would find in a journal article. Be sure to provide a table of results
AND a written summary of the results in your
response.}\label{repeat-the-analysis-you-ran-in-question-3-in-which-you-predict-absences-from-empsatisfaction-engagementsurvey-and-tenure-but-first-control-for-sex.-provide-a-summary-of-this-analysis-like-what-you-would-find-in-a-journal-article.-be-sure-to-provide-a-table-of-results-and-a-written-summary-of-the-results-in-your-response.}

\begin{Shaded}
\begin{Highlighting}[]
\CommentTok{\# Fit the multiple regression model with Sex as a covariate}
\NormalTok{model\_step1\_sex }\OtherTok{\textless{}{-}} \FunctionTok{lm}\NormalTok{(Absences }\SpecialCharTok{\textasciitilde{}}\NormalTok{ Sex, }\AttributeTok{data =}\NormalTok{ HRData)}
\NormalTok{model\_step2\_sex }\OtherTok{\textless{}{-}} \FunctionTok{lm}\NormalTok{(Absences }\SpecialCharTok{\textasciitilde{}}\NormalTok{ Sex }\SpecialCharTok{+}\NormalTok{ EmpSatisfaction }\SpecialCharTok{+}\NormalTok{ EngagementSurvey }\SpecialCharTok{+}\NormalTok{ Tenure, }\AttributeTok{data =}\NormalTok{ HRData)}

\CommentTok{\# Summary of the models}
\NormalTok{summary\_step1\_sex }\OtherTok{\textless{}{-}} \FunctionTok{summary}\NormalTok{(model\_step1\_sex)}
\NormalTok{summary\_step2\_sex }\OtherTok{\textless{}{-}} \FunctionTok{summary}\NormalTok{(model\_step2\_sex)}
\end{Highlighting}
\end{Shaded}

\begin{Shaded}
\begin{Highlighting}[]
\CommentTok{\# Create tables of results}
\NormalTok{results\_step1\_sex }\OtherTok{\textless{}{-}} \FunctionTok{as.data.frame}\NormalTok{(summary\_step1\_sex}\SpecialCharTok{$}\NormalTok{coefficients)}
\FunctionTok{colnames}\NormalTok{(results\_step1\_sex) }\OtherTok{\textless{}{-}} \FunctionTok{c}\NormalTok{(}\StringTok{"Estimate"}\NormalTok{, }\StringTok{"Std. Error"}\NormalTok{, }\StringTok{"t value"}\NormalTok{, }\StringTok{"Pr(\textgreater{}|t|)"}\NormalTok{)}

\NormalTok{results\_step2\_sex }\OtherTok{\textless{}{-}} \FunctionTok{as.data.frame}\NormalTok{(summary\_step2\_sex}\SpecialCharTok{$}\NormalTok{coefficients)}
\FunctionTok{colnames}\NormalTok{(results\_step2\_sex) }\OtherTok{\textless{}{-}} \FunctionTok{c}\NormalTok{(}\StringTok{"Estimate"}\NormalTok{, }\StringTok{"Std. Error"}\NormalTok{, }\StringTok{"t value"}\NormalTok{, }\StringTok{"Pr(\textgreater{}|t|)"}\NormalTok{)}

\CommentTok{\# Print the tables of results}
\FunctionTok{print}\NormalTok{(results\_step1\_sex)}
\end{Highlighting}
\end{Shaded}

\begin{verbatim}
##                Estimate Std. Error    t value     Pr(>|t|)
## (Intercept) 10.26136364  0.4418647 23.2228661 9.811726e-70
## SexM        -0.05395623  0.6706603 -0.0804524 9.359295e-01
\end{verbatim}

\begin{Shaded}
\begin{Highlighting}[]
\FunctionTok{print}\NormalTok{(results\_step2\_sex)}
\end{Highlighting}
\end{Shaded}

\begin{verbatim}
##                       Estimate   Std. Error      t value     Pr(>|t|)
## (Intercept)       9.7222014520 2.1637308074  4.493258320 9.957754e-06
## SexM             -0.0032025272 0.6696150774 -0.004782639 9.961871e-01
## EmpSatisfaction   0.5135414582 0.3718800420  1.380933097 1.683072e-01
## EngagementSurvey -0.1628279453 0.4279651238 -0.380470128 7.038604e-01
## Tenure           -0.0006058126 0.0003499433 -1.731173725 8.442874e-02
\end{verbatim}

\begin{Shaded}
\begin{Highlighting}[]
\CommentTok{\# Calculate variance explained by Sex in Step 1}
\NormalTok{variance\_explained\_step1\_sex }\OtherTok{\textless{}{-}}\NormalTok{ summary\_step1\_sex}\SpecialCharTok{$}\NormalTok{r.squared }\SpecialCharTok{*} \DecValTok{100}

\CommentTok{\# Calculate change in R{-}square from Step 1 to Step 2}
\NormalTok{change\_in\_r\_squared\_sex }\OtherTok{\textless{}{-}}\NormalTok{ summary\_step2\_sex}\SpecialCharTok{$}\NormalTok{r.squared }\SpecialCharTok{{-}}\NormalTok{ summary\_step1\_sex}\SpecialCharTok{$}\NormalTok{r.squared}

\CommentTok{\# Calculate percentage of variance explained by the full model}
\NormalTok{variance\_explained\_full\_model\_sex }\OtherTok{\textless{}{-}}\NormalTok{ summary\_step2\_sex}\SpecialCharTok{$}\NormalTok{r.squared }\SpecialCharTok{*} \DecValTok{100}
\end{Highlighting}
\end{Shaded}

\subsection{Summary}\label{summary-3}

Low R-squared values signal limited explanatory power of the predictors.
The small variance explained in Step 1 (by Sex) and the small increase
in Step 2 shows the need for more significant predictors.

None of the predictors in the Full Model were statistically significant,
showing the importance of assessing each predictor's contribution to the
model. This shows the importance of revisiting variable selection,
exploring potential interactions, and any omitted predictors.

Although the model was statistically significant overall, its practical
utility is limited due to the low variance explained as you have to
distinguish statistical significance from practical significance.

The model's insights can guide further exploration into factors
influencing Absences. The analysis highlights the need for additional
predictors or a refined hypotheses to better capture the variability in
this outcome.

\subsection{Answers a, b, and c.}\label{answers-a-b-and-c.-3}

\subsubsection{a. How much variance did Sex explain as a covariate in
Step 1 of your
Model?}\label{a.-how-much-variance-did-sex-explain-as-a-covariate-in-step-1-of-your-model}

In Step 1, Sex was used as a single covariate to predict Absences. The
model explained only 0.0021\% of the variance in Absences (R² =
0.000021), which is negligible.

The model's F-statistic was F(1, 309) = 0.00647, p \textgreater{} 0.05,
showing that Sex alone is not a significant predictor of Absences. While
including a covariate like Sex can help adjust the model, its
contribution to explaining variance must be assessed.

\subsubsection{b. What is the change in R-square when you go from Step 1
to Step
2?}\label{b.-what-is-the-change-in-r-square-when-you-go-from-step-1-to-step-2-1}

When additional predictors (EmpSatisfaction, EngagementSurvey, and
Tenure) were added in Step 2, the R-squared increased from 0.000021 in
Step 1 to 0.0158 in Step 2.

This represents a change in R-squared of 0.0158, or 1.58\% additional
variance explained by the inclusion of the three predictors. While this
improvement is modest, it highlights the potential benefit of adding
multiple predictors, even when individual predictors are not
statistically significant showing the importance of cumulative effect in
evaluating model fit enhancements.

\subsubsection{c.~What percentage of variance in Absences is explained
by the Full
Model?}\label{c.-what-percentage-of-variance-in-absences-is-explained-by-the-full-model}

The Full Model (Step 2) explained 1.58\% of the variance in Absences (R²
= 0.0158). Although statistically significant (F(4, 306) = 1.231, p
\textless{} 0.05), this low R-squared value indicates that most of the
variability in Absences remains unexplained by the predictors included
in the model.

\end{document}
